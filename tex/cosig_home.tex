\documentclass[letterpaper, 12pt]{article}
\input{tex/_preamble}
\begin{document}
\flushleft\includegraphics[width=\textwidth]{img/home/241017_final_logo_mockup.png}

\section*{Anyone can do post-publication peer review.}
\section*{Anyone can be a steward of the scientific literature.}
\section*{Anyone can do forensic metascience.}
\section*{Anyone can sleuth.}

However, investigating the integrity of the published scientific literature often requires domain-specific knowledge that not everyone will have. This open source project is a collection of guides written and maintained by publication integrity experts to distribute this domain-specific knowledge so that others can participate in post-publication peer review.

COSIG currently hosts 29 guides and was last updated on 24 June 2025.

Suggestions to improve COSIG can be submitted by opening an issue on \href{https://github.com/cosig-pppr/cosig/issues}{COSIG's GitHub repo} or by emailing \href{mailto:admin@cosig.net}{admin@cosig.net}. Before contributing, read COSIG's \href{https://github.com/cosig-pppr/cosig/blob/main/CONTRIBUTING.md}{Contributing} and \href{https://github.com/cosig-pppr/cosig/blob/main/CODE_OF_CONDUCT.md}{Code of Conduct} pages.

Except where otherwise indicated, all material in COSIG is available under a \href{https://creativecommons.org/licenses/by-nc-sa/4.0/deed.en}{CC BY-NC-SA 4.0 license}. That means that you are free to distribute, remix, adapt, and build upon the material in any medium or format for noncommercial purposes only, and only so long as COSIG is properly cited. If you remix, adapt, or build upon the material, you must license the modified material under identical terms.

COSIG itself can be cited as:

\begin{quote}
    (2025, June 4). Collection of Open Science Integrity Guides. Retrieved from \href{https://doi.org/10.17605/OSF.IO/2KDEZ}{https://doi.org/10.17605/OSF.IO/2KDEZ}
\end{quote}

\href{https://doi.org/10.5281/zenodo.15564777}{A commentary} explaining the motivation behind COSIG can be cited as:
\begin{quote}
    Richardson, R. (2025). The Collection of Open Science Integrity Guides (COSIG): Expanding participation in post-publication peer review. Zenodo. \href{https://doi.org/10.5281/zenodo.15564777}{https://doi.org/10.5281/zenodo.15564777}
\end{quote}

\includegraphics[width=0.2\textwidth]{img/home/Cc-by-nc-sa_icon.svg.png}

\pagebreak

The following individuals have contributed to COSIG in some way:

\begin{itemize}
    \setlength\itemsep{-0.5em}
    \item Anna Abalkina \orcidlink{0000-0003-1469-4907}
    \item Ren\'e Aquarius \orcidlink{0000-0002-0968-6884}
    \item Lonni Besan\c{c}on \orcidlink{0000-0002-7207-1276}
    \item Elisabeth Bik \orcidlink{0000-0001-5477-0324}
    \item David Bimler \orcidlink{0000-0003-2405-0697}
    \item Jennifer Byrne \orcidlink{0000-0002-8923-0587}
    \item Guillaume Cabanac \orcidlink{0000-0003-3060-6241}
    \item Jana Christopher \orcidlink{0000-0002-2699-3368}
    \item M.V. Dougherty \orcidlink{0000-0002-8731-6045}
    \item Ian Hussey \orcidlink{0000-0001-8906-7559}
    \item Yagmur Ozturk \orcidlink{0000-0002-2843-8990} (\textbf{Maintainer})
    \item Kevin Patrick \orcidlink{0009-0001-7491-9242}
    \item Solal Pirelli \orcidlink{0009-0003-4336-1316} (\textbf{Maintainer})
    \item Reese Richardson \orcidlink{0000-0002-6058-5886} (\textbf{Maintainer})
    \item Nicholas Ritchie \orcidlink{0000-0001-5734-5729}
    \item Matt Spick \orcidlink{0000-0002-9417-6511}
    \item Stefan Stender \orcidlink{0000-0003-0281-5900}
    \item \textit{Nerita vitiensis} (pseudonymous)
\end{itemize}

\end{document}
