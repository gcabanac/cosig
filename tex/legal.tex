\documentclass[letterpaper, 12pt]{article}
\usepackage{geometry}
\geometry{
    letterpaper,
    left=20mm,
    top=20mm,
    bottom=20mm
}
\usepackage{tocloft}
\usepackage{graphicx}
\usepackage{authblk}
\usepackage{amssymb}
\usepackage{lipsum}
\usepackage{float}
\usepackage{times}
\usepackage{amsmath}
\usepackage[format=plain,
            labelfont={bf,it},
            textfont=it]{caption}
\captionsetup{justification=raggedright,singlelinecheck=false}
\usepackage{ragged2e}
\usepackage{longtable}
\usepackage{comment}
\usepackage{setspace}
\usepackage{fancyhdr}
\usepackage{titlesec}
\usepackage[hyperindex,breaklinks]{hyperref}
\hypersetup{
    colorlinks=true,
    linkcolor=blue,
    filecolor=magenta,      
    urlcolor=blue
}
\usepackage[T1]{fontenc}
\usepackage{helvet}
\renewcommand{\familydefault}{\sfdefault}
\pagenumbering{gobble}
\usepackage[skip=10pt plus1pt, indent=40pt]{parskip}
\usepackage{orcidlink}
\usepackage{standalone}

\titlespacing*{\section}
{0pt}{1.5ex plus 1ex minus .2ex}{1.3ex plus .2ex}

\renewcommand\Authfont{\fontsize{12}{14.4}\selectfont}
\renewcommand\Affilfont{\fontsize{9}{10.8}\itshape}

\newcommand{\cosigpart}[1]{
  \addcontentsline{toc}{part}{#1}
}
\newcommand{\cosigsection}[1]{
  \section*{#1}
  \phantomsection % avoid warnings from hyperref about the anchor of a bookmark and its parent's
  \addcontentsline{toc}{section}{#1}
}
\begin{document}
\flushleft\includegraphics[width=0.5\textwidth]{img/home/241017_final_logo_mockup.png}

\cosigsection{Legal considerations}
\textit{Last updated: 20 June 2025}

Anything you write online can be used against you.
This guide will help you understand and lower that risk.
\textbf{This is not legal advice, if you want to know the exact situation in your jurisdiction please consult a lawyer.}

\subsection*{Laws and jurisdictions}

Laws are typically written in language that looks relatively straightforward, even if a bit verbose. This does not mean they are straightforward to understand. Laws frequently use legal ``terms of art'' whose precise meaning is defined elsewhere, or even not fully defined and only clarified through the jurisprudence of past rulings.

Furthermore, everybody falls under a set of jurisdictions, such as countries, sub-national divisions like states, and supra-national groups like the European Union. Generally speaking, ``lower'' jurisdictions can only make the law more restrictive, e.g., a city cannot give you the right to do something that is illegal in its country.

While you should be able to easily find which jurisdictions you are in, this is where the simple part ends.
Laws frequently clash in subtle ways that aren't reasonable to understand on your own.
Some laws in one jurisdiction may even be entirely unenforceable due to laws in a ``higher'' jurisdiction that supersede them.

In summary, \textbf{do not assume that you can fully understand legal risks just because you've read the text of a law}.
The meaning may differ from what you think, some parts of the law may be unenforceable, and relevant information could be missing entirely because it is defined in another text ``higher up'' the jurisdiction chain. There's a reason lawyers exist and are paid well.

\subsection*{Lawsuits and threats}

Anyone can try to sue anyone else for any reason.
Without concrete evidence, a lawsuit will typically be dropped before you need to worry about it, possibly without you even knowing about it.
However, understand that you cannot generally prevent someone from attempting to sue you, unless you post entirely anonymously using services that refuse to cooperate with lawsuits, such as PubPeer.

Most threats of the form ``I will sue you unless...'' go nowhere, because the person making the threats does not actually have a case, or is unwilling to spend the amount of money required to pay a lawyer given the usually low probability they will win.
This does not mean you should entirely ignore such letters, as they occasionally do lead to lawsuits.
Think about your situation and weigh it against what you are being asked to do seriously.

Lawsuits purely designed to silence critics are known as \href{https://en.wikipedia.org/wiki/Strategic_lawsuit_against_public_participation}{SLAPP},
``Strategic Lawsuits Against Public Participation''.
In jurisdictions where even simple lawsuits are expensive to file and defend against, the goal may not even be to win, but rather to cause the critic to give up rather than pay lots of money.

Some jurisdictions have ``anti-SLAPP'' laws, which allow a defendant to quickly force the plaintiff to prove their case. These are intended to minimize the time and money a defendant must spend on a SLAPP lawsuit. However, not all jurisdictions have this, so do not assume you have such a tool unless you've checked.

\subsection*{Institutional backing}

If you are currently working somewhere, especially a research facility such as a university, your employer may or may not back you.
This may or may not match whatever public stance they take around ``open science'' and the importance of reproducibility.
For a particularly negative case, see \href{https://www.science.org/content/article/paul-brookes-surviving-outed-whistleblower}{Paul Brookes}, whose university was not supportive of his ``science-fraud.org'' blog.

\subsection*{Words matter}

Regardless of where you're browsing the Internet from, you've probably been bathed in US-centric jargon, such as ``First Amendment'' rights, legal myths around ``shouting fire in a crowded theater'', and standards such as ``actual malice''.

Many places do \emph{not} have speech protections as strong as the United States'. For instance, many countries criminalize ``hate speech'' and have relatively low standards for what consists as defamation. \textbf{Learn how defamation / libel / slander / ... work in your country, do not assume that what you've read on the Internet applies everywhere.}

It may be tempting to imitate the snarky and brutal tone you can read in some websites, but depending on where you live that could lead you to legal trouble. For instance:
\begin{itemize}
    \item Speech that derides groups of people, especially by ethnicity, religion, or other common ``protected classes'', can be seen as hate speech in many jurisdictions. \\
    For instance, if you've busted a citation ring with authors from the same country, you probably want to avoid any snarky remarks about what this could mean for that country. It's also just generally not a nice thing to do.
    \item Accusing someone of a concrete malicious act, especially if that act's name corresponds to a crime in your jurisdiction, can be seen as defamatory. \\
    This is why most comments around scientific fraud focuses on facts, such as what is concretely in a paper, rather than attributing intent to authors.
    \item Speech that deliberately demeans someone without a clear reason, especially if the same effect could have been achieved with less drama, can be seen as defamatory. \\
    Such speech is also likely to lessen the impact of what you're saying, since it makes you look like a bully, so just don't do it.
    \item \textbf{Truth is not always a defense} to claims of defamation, depending on your jurisdiction and the exact context. \\
    Similarly, ``good faith'' is not always a defense.
\end{itemize}

\subsection*{Lawsuit examples}

In the US, Francesca Gino was a Harvard professor accused of research misconduct by a group of researchers under the name ``Data Colada''.
She was suspended then fired by Harvard. She tried suing both Data Colada and Harvard, but \href{https://reason.com/volokh/2024/09/11/prof-francesca-ginos-libel-claims-against-harvard-business-school-and-data-colada-dismissed/}{lost}.
However, this cost the Data Colada group \href{https://www.vox.com/future-perfect/23841742/francesca-gino-data-colada-lawsuit-gofundme-science-culture-transparency-academic-fraud-dishonesty}{hundreds of thousands of dollars} in legal fees.

Still in the US, cancer researcher Carlo Croce sued researcher David Sanders who had been interviewed by the New York Times around falsified data in Croce's papers. Croce \href{https://retractionwatch.com/2020/05/13/cancer-researcher-loses-defamation-suit-against-critic/}{lost}, but the judge did not grant Sanders the benefit of Indiana's ``anti-SLAPP'' law either.

In the UK, businessman Arron Banks \href{https://www.bbc.com/news/uk-65644475}{sued} journalist Carole Cadwalladr due to her accusing him of lying about ``his relationship with the Russian government''.
After appealing, Banks' claims were partly upheld, and Cadwalladr has been ordered to pay £1.2m.
As of the time of writing this guide, this case is still under appeal at the level of the European Court of Human Rights.

\end{document}