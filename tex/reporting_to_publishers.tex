\documentclass[letterpaper, 12pt]{article}
\input{tex/_preamble}
\begin{document}
\flushleft\includegraphics[width=0.5\textwidth]{img/home/241017_final_logo_mockup.png}

\cosigsection{Reporting publication integrity issues to publishers}
\textit{Last updated: 8 July 2025}

\subsection*{Integrity/ethics teams at publishers}

Most large scientific publishers have whole departments for publication ethics and publication integrity. These departments investigate incoming reports of possible errors or misconduct and will advise editors and editors-in-chief on how to correct the scientific record. Usually these departments will follow the guidelines set by the \href{https://publicationethics.org/}{Committee on Publication Ethics (COPE)}, which generally consists of reviewing the evidence, contacting the journal and authors and coming to some sort of closure.

\subsection*{How to report issues to publishers}

The best way to reach these teams is by email. You can send this email both to the publisher's integrity team as well as to the journal's editors-in-chief. Some journals will have editorial board members that specifically handle ethics concerns or certain topics (e.g., disciplinary fields or specific methodologies).

Just as described in the guide for \href{https://osf.io/sghaq}{PubPeer commenting best practices}, you should be clear and neutral when describing potential publication integrity issues to these teams. Always include the \href{https://www.doi.org/}{DOI} (or other permanent identifier, such as PubMed ID) of the article(s) in question and report in a neutral way about the issue(s). Including a link to PubPeer where the issues are discussed in greater depth or an including an image to illustrate an issue may be useful.

Always be respectful when you contact publishers' integrity teams. Do not project frustrations or anger onto the members of these teams. Feel free to ask for a confirmation of receipt at the end of your email in order to make sure that the recipient has received all the relevant information.

\subsection*{Example of a helpful email to a publisher}

\begin{quote}
\textit{To whom it may concern,}\\
\indent\\
\textit{We have detected duplicated images in figures of articles published in [journal].}\\
\indent\\
\textit{We found an overlap between figure 1, panel A of Article 1 [DOI] and Figure 5, panel B of Article 2 [DOI]. We also found an overlap between Figure 3, panel C of Article 1 [DOI] and Figure 1, panel E of Article 3 [DOI].}\\
\indent\\
\textit{All of our findings have also been made publicly available on PubPeer:}\\
\indent\\
\textit{Article 1: [PubPeer link]\\
Article 2: [PubPeer link]\\
Article 3: [PubPeer link]\\}
\indent\\
\textit{Feel free to contact us with your questions. We’re happy to clarify any issues you might have. Could all of the recipients send us a confirmation of receipt of this email and let us know what the timeframe will be for assessing these issues?}\\
\indent
\textit{Best,}\\
\indent\\
\textit{[Name/pseudonym]}
\end{quote}

\subsection*{What happens next?}

You should be very patient after reporting an issue; investigations often take months if not years to resolve, even for blatant issues that seem to warrant immediate action. A publisher's integrity team may take one of several actions after being contacted regarding a publication integrity or ethics issue:

\begin{itemize}
    \setlength\itemsep{-0.5em}
    \item When an investigation finds that the conclusions of the research are not affected, the authors will be given the chance to clarify or correct their article post-publication. This will be often done in the form of a \textit{correction}, which should detail what the initial problem was and how the problem was resolved.
    \item When an investigation finds that conclusions of the research are affected or the work cannot be trusted, the article will usually be \textit{retracted}. A retraction note will be published, detailing why the problem was too substantial to correct.
    \item When it is unclear whether the conclusions are affected by the reported issue, an \textit{expression of concern} might be issued. This is a way for publishers to show that readers should exercise increased caution when interpreting the results of the published article.
    \item Nothing happens. Research integrity teams handle many cases and may drop a case with or without notifying the party that raised the issue.
\end{itemize}

When reporting an issue, you may ask to be notified when an investigation is resolved or an editorial notice is applied to an article. COPE \href{https://doi.org/10.24318/cope.2019.2.25}{recommends} that editors inform the person that originally raised concerns when the an outcome is reached, but many publishers fail to adhere to this guidance.

\pagebreak
\subsection*{Contact information for integrity/ethics teams at major publishers}

\begin{itemize}
    \setlength\itemsep{-0.5em}
    \item American Association for the Advancement of Science (AAAS, \textit{Science}, \textit{Science Advances}, etc.):  science\_data@aaas.org.
    \item American Chemical Society (ACS): Each journal usually has contact information for a managing editor, usually at managing.editor@<journal-url>.org
    \item American Society for Microbiology: ethics.journals@asmusa.org
    \item Association for Computer Machinery: \url{https://services.acm.org/ethics/report.cfm}
    \item British Medical Journal (BMJ): publication.ethics@bmj.com
    \item Cambridge University Press (CUP): publishingethics@cambridge.org
    \item Elsevier (Elsevier, Cell Press, etc.): ethicsexpert@elsevier.com
    \item Frontiers: research.integrity@frontiersin.org
%    \item Hindawi: publication.ethics@hindawi.com
    \item Institute of Electrical and Electronics Engineers (IEEE): pub-ethics@ieee.org
    \item Karger: publication.ethics@karger.com
    \item MDPI: publication.ethics@mdpi.com
    \item Oxford University Press (OUP): journals.ethics@oup.com
    \item PLOS: pub-ethics@plos.org
    \item Royal Society of Chemistry (RCS): publishingethics@rsc.org
    \item Rockefeller University Press (\textit{Journal of Cell Biology}, \textit{Life Science Alliance}, etc.): integrity@rupress.org
    \item Sage (Sage, Mary-Ann Liebert): publication\_ethics@sagepub.com
    \item Springer Nature (Springer, Nature, BMC): ethics.reporting@springernature.com
    \item Taylor \& Francis (Taylor \& Francis, Dove Medical Press, CRC Press, Routledge): ethics@tandf.co.uk
    \item The Institution of Engineering and Technology (IET): ethics@theiet.org
    \item Thieme: publishingethics@thieme.de
    \item Wiley (Wiley, Hindawi, FEBS Press): researchintegrity@wiley.com
    \item Wolters Kluwer: customer.service@wolterskluwer.com and \url{https://www.wolterskluwer.com/en/about-us/ethics-and-compliance}
\end{itemize}

\subsection*{Additional resources}

\begin{itemize}
    \setlength\itemsep{-0.5em}
    \item \href{https://doi.org/10.3145/epi.2023.ene.18}{``How do journals deal with problematic articles. Editorial response of journals to articles commented in PubPeer'' (2023)}
\end{itemize}

\end{document}